\documentclass[12pt]{article}
\usepackage[paper=letterpaper,margin=2cm]{geometry}
\usepackage{amsmath}
\usepackage{amssymb}
\usepackage{amsfonts}
\usepackage{newtxtext, newtxmath}
\usepackage{enumitem}
\usepackage{titling}
\usepackage[colorlinks=true]{hyperref}

\setlength{\droptitle}{-6em}

\title{Assignment \#1}
\author{Aidan Traboulay}
\date{\today}

\begin{document}
\maketitle
\begin{enumerate}[leftmargin=\labelsep]
\item In each of the following cases, find a proposition with the given conditions.
    \begin{enumerate}
    \item Find a proposition with three variables p, q, and r that is true when p and r are true and q is false, and false otherwise.
    \begin{equation*}
        \left(p \wedge r \wedge \lnot q \right)
    \end{equation*}
    \item Find a proposition with three variables p, q, and r that is true when at most one of the three variables is true, and false otherwise.
    \begin{equation*}
        \left(\lnot p \wedge \lnot q \wedge \lnot r \right) \vee \left(\lnot p \wedge q \wedge r\right) \vee \left(p \wedge \lnot q \wedge r \right) \vee \left(p \wedge q \wedge \lnot r\right)
    \end{equation*}

    \item Find a proposition with three variables p, q, and r that is never true.
        \begin{equation*}
            \left(p \wedge \lnot p\right) \vee \left(q \wedge \lnot q\right) \vee \left(r \wedge \lnot r\right)
        \end{equation*}
    \end{enumerate}

\item In each of the following cases, write an equivalent proposition as stated.
    \begin{enumerate}
    \item Write a proposition equivalent to $ p \vee \lnot q $ that uses only p, q, $\lnot$, and the connective $\wedge$.
    \begin{equation*}
        p \vee \lnot q \equiv \lnot \left(\lnot p \wedge q\right)
    \end{equation*}
    \item Write a proposition equivalent to $ p \to q $ using only p, q, $ \lnot $ , and the connective $\vee.$
    \begin{equation*}
        p \to q \equiv \lnot p \vee q
    \end{equation*}
    \end{enumerate}
    
\item Prove that $\left(q \wedge \left(p \to \lnot q\right)\right) \to \lnot p$ is a tautology using propositional equivalence and the laws of logic.

    Since $ a \to b \equiv \lnot a \vee b $,  $p \to \lnot q \equiv \lnot p \vee \lnot q $
    \begin{equation*}
        q \wedge \left(\lnot p \vee \lnot q\right) \to \lnot p
    \end{equation*}
    \begin{equation*}
        \left(q \wedge \lnot p\right) \vee \left(q \wedge \lnot q\right) \to \lnot p 
    \end{equation*}
    \begin{equation*}
        \left(q \wedge \lnot p\right) \to \lnot p 
    \end{equation*}
    \begin{equation*}
        \lnot \left(q \wedge \lnot p\right) \vee \lnot p
    \end{equation*}
    \begin{equation*}
        \left(\lnot q \vee \lnot \left(\lnot p \right) \right) \vee \lnot p
    \end{equation*}
    \begin{equation*}
         \lnot q \vee p \vee \lnot p
    \end{equation*}
    \begin{equation*}
        \lnot q \vee T
    \end{equation*}
    
\item Using c for “it is cold”, d for “it is dry”, r for “it is rainy”, and w for “it is windy”, write the following sentences in symbols.
    \begin{enumerate}
        \item It is neither cold nor dry.
            \begin{equation*}
                \lnot c \wedge \lnot d
            \end{equation*}
        \item It is rainy if it is not cold.
            \begin{equation*}
                \lnot c \to r
            \end{equation*}
        \item To be windy it is necessary that it be cold
            \begin{equation*}
                w \to c
            \end{equation*}
        \item It is rainy only if it is windy and cold.
            \begin{equation*}
                r \to \left(w \wedge c\right)
            \end{equation*}
    \end{enumerate}

\item There are three kinds of people living on an island: knights who always tell
the truth, knaves who always lie, and spies who can either tell the truth or lie. You encounter
three people, A, B, and C. You know one of the three people is a knight, one is a knave,
and one is a spy. Each of the three people knows the type of person each of the other two is.
How would you identify who is who, if:
    \begin{enumerate}
        \item A says “I am not a knight,” B says “I am not a spy,” and C says “I am not a knave.”
            \begin{tabbing}
                \> Knights always tell the truth so A can not be the Knight and it can not be a Knave as they always lie.
            \end{tabbing}
        \item A says “I am a spy,” B says “I am a spy” and C says “B is a spy.”
            \begin{tabbing}
                \> A is the Knave \\
                \> B is the spy \\
                \> C is the Knight
            \end{tabbing}
    \end{enumerate}
\item Determine whether the following argument is valid. Justify your answer by showing which rules of inferences have been followed at each step.
    \begin{tabbing}
        \> She is a Math Major or a Computer Science Major \\
        \> If she does not know discrete math, she is not a Math Major. \\
        \> If she knows discrete math, she is smart. \\
        \> She is not a Computer Science Major. \\
        \> Therefore, she is smart.
    \end{tabbing}
    
    M $\simeq$ She is a math major\\
    C $\simeq$ She is a computer science major \\
    D $\simeq$ She knows discrete math \\
    S $\simeq$ She is smart \\
    \begin{equation*}
        M \vee C
    \end{equation*}
    \begin{equation*}
       \lnot D \to \lnot M
    \end{equation*}
    \begin{equation*}
        D \to S
    \end{equation*}    
    \begin{equation*}
        \lnot C
    \end{equation*}    
    \begin{equation*}
       \therefore S
    \end{equation*}
    \therefore $ The argument is valid $

\item Give a direct proof of the theorem: “if x and y are odd integers, then x + y is even.”
    \begin{center}
        p : x is odd \\
        q: y is odd \\
        r: x + y is even 
    \begin{equation*}
        \therefore \left(p \wedge q\right) \to r
    \end{equation*}
    Contrapositve statement: $\lnot r \to \lnot\left(p \wedge q \right)$ \\
    Using De Morgan's law: $\lnot r \to \left(\lnot p \vee \lnot q \right)$ \\
    So, $\lnot r \to \left(\lnot p \vee \lnot q \right)$ states "if x + y is not even then either p is not odd or q is not odd." \\
    Thus, $\lnot r \to \left(\lnot p \vee \lnot q \right)$ states "if x + y is odd then either p is even or q is even."
\end{center}  
\item Give a proof by contradiction of the following: “If x and y are even integers, then
xy is even”.

\end{enumerate}
\end{document}
