\documentclass[12pt]{article}
\usepackage[paper=letterpaper,margin=2cm]{geometry}
\usepackage{amsmath}
\usepackage{amssymb}
\usepackage{amsfonts}
\usepackage{newtxtext, newtxmath}
\usepackage{enumitem}
\usepackage{titling}
\usepackage[colorlinks=true]{hyperref}

\setlength{\droptitle}{-6em}

\title{Assignment \#2\\
        \large{CP214}
      }

\author{Aidan Traboulay}
\date{\today}

\begin{document}
\maketitle
\begin{enumerate}[leftmargin=\labelsep]

\item Prove that $ A \cap (B \cup C) = (A \cap B) \cup (A \cap C)$ by giving a containment proof (that is, prove that the left side is a subset of the right side and that the right side is a subset of the left side).
    \begin{flalign}\hspace{4em} 
        (LHS) \nonumber\hspace{0.5em} Let \hspace{0.5em}  & x \in A \cap (B \cup C) & \\ 
        & \nonumber(x \in A) \wedge (x \in B \cup C) & \\ 
        & \nonumber(x \in A) \wedge (x \in B \vee x \in C) & \\
        & \nonumber(x \in A \wedge x \in B) \vee (x \in A \wedge x \in C) & \\ 
        & \nonumber(x \in A \cap B) \vee (x \in A \cap C) & \\ 
        & \nonumber x \in (A \cap B) \cup (A \cap C) & \\
        \therefore \hspace{0.5em} & A \cap (B \cup C) \subseteq (A \cap B) \cup (A \cap C) & \\
        \newline
        (RHS) \nonumber\hspace{0.5em} Let \hspace{0.5em} & x \in (A \cap B) \cup (A \cap C) & \\
        & \nonumber(x \in A \cap B) \cup (x \in A \cap C) & \\
        & \nonumber(x \in A \wedge x \in B) \vee (x \in A \wedge x \in C) & \\ 
        & \nonumber(x \in A) \wedge (x \in B \vee x \in C) & \\
        & \nonumber(x \in A) \wedge (x \in B \cup C) & \\
        & \nonumber x \in A \cap (B \cup C) & \\
        \therefore \hspace{0.5em} & (A \cap B) \cup (A \cap C) \subseteq A \cap (B \cup C) & \\
        LHS = RHS
    \end{flalign}

\item Prove that $ A \cap (B \cup C) = (A\cap B) \cup (A \cap C)  $ by giving a proof using logical equivalence.
    \begin{flalign}\hspace{4em}
        \nonumber & A \cap (B \cup C) = (A \cap B) \cup (A \cap C) & \\
        \nonumber & x \in [A \cap (B \cup C)] & \\
        \nonumber & (x \in A \wedge x \in B) \vee (x \in A \wedge x \in C) & \\
        \nonumber & x \in (A \cap B) \vee x \in (A \cap C) & \\
        & x \in [(A \cap B) \cup (A \cap C)]
    \end{flalign}

\item Suppose $ U = \{1,2,...,9\}$, $ A = $ all multiples of 2, $ B = $ all multiples of 3, and $ C = \{3,4,5,6,7\} $. Find $ C - (B - A)$.
    \begin{flalign}\hspace{4em}
        \nonumber & A = \{2,4,6,8\} & \\
        \nonumber & B = \{3,6,9\} & \\
        \nonumber & C = \{3,4,5,6,7\} & \\
        &  C - (B-A) = \{3,4,5,6,7\} - [\{3,6,9\} - \{2,4,6,8\}] = \{4,5,6,7\}
    \end{flalign}

\item Suppose $ f: \textbf{N} \to \textbf{N} $ has the rule $ f(n) = 4n + 1 $.
    \begin{enumerate}[label=(\roman*)]
    \item Determine whether $ f $ is 1-1. Justify your answer.
        \begin{flalign}\hspace{4em}
            \nonumber Let \hspace{0.5em} & f(n) = 4n + 1 & \\
            \nonumber & f(m) = 4y + 1 & \\
            \nonumber & f(n) = f(m) & \\
            \nonumber & 4n + 1 = 4m + 1 & \\
            \nonumber & \frac{4n}{4} = \frac{4m}{4} & \\
            \therefore \hspace{0.5em} & n = m \to f(n) = f(m) \to 1 - 1& 
        \end{flalign}

    \item Determine whether $ f $ is onto \textbf{N}. Justify your answer.
        \begin{flalign}\hspace{4em}
            \nonumber Let \hspace{0.5em} & f(n) = 4n + 1 & \\
            \nonumber & 1 = 4n + 1 & \\
            \nonumber & 0 = 4n & \\
            \nonumber & 0 = n& \\
            \nonumber & 0 \notin N & \\
            \therefore & \hspace{0.5em}\in \hspace{0.5em} \text{is not onto because 1 is not in the range.} &
        \end{flalign}
    \end{enumerate}
    
\item Suppose $ f: \textbf{R} \to \textbf{Z} $ has the rule $ f(x) = \lceil2x - 1\rceil $.
    \begin{enumerate}[label=(\roman*)]
        \item If $ A = \{x \hspace{0.3em} \vert \hspace{0.3em} 1 \leq x \leq 4\}$, find $ f(A)$.
            \begin{flalign}\hspace{4em}
                \nonumber \hspace{0.5em} & f(A) = \{ \lceil2.1 - 1\rceil, \lceil2.2 - 1\rceil, \lceil2.3 - 1\rceil, \lceil2.4 - 1\rceil \} &\\
                \therefore \hspace{0.5em} & f(A) = \{1,3,5,7\}& 
            \end{flalign}
            
        \item If $ B = \{3,4,5,6,7\}$, find $ f(B)$.
            \begin{flalign}\hspace{4em}
                \nonumber \hspace{0.5em} & f(B) = \{ \lceil2.3 - 1\rceil, \lceil2.4 - 1\rceil, \lceil2.5 - 1\rceil, \lceil2.6 - 1\rceil \lceil2.7 - 1\rceil\} &\\
                \therefore \hspace{0.5em} & f(B) = \{5,7,9,11,13\}& 
            \end{flalign}        
            
        \item If $ C = \{-9,-8\}$, find $ f^{-1}(C)$.
            \begin{flalign}\hspace{4em}
                \nonumber & \in: R \to Z & \\
                \nonumber & \in^{-1}: Z \to R & \\ 
                \nonumber & f^{-1}(x) = 0.5(x + 1) & \\
                \nonumber \hspace{0.5em} & f(C) = \{ \lceil0.5(-9 + 1)\rceil, \lceil0.5(-8 + 1)\rceil\} = \{-4, -3.5\}& \\
                \therefore \hspace{0.5em} & f^{-1}(C) = \{-4,-3\} & 
            \end{flalign}        
            
        \item If $ D = \{0.4,0.5,0.6\}$, find $ f^{-1}(D)$.
            \begin{flalign}\hspace{4em}
                \nonumber & \in: R \to Z & \\
                \nonumber & \in^{-1}: Z \to R & \\ 
                \nonumber & f^{-1}(x) = 0.5(x + 1) & \\
                \nonumber \hspace{0.5em} & f(D) = \{ \lceil0.5(0.4 + 1)\rceil, \lceil0.5(0.5 + 1)\rceil, \lceil0.5(0.6 + 1)\rceil\} = \{\lceil0.7\rceil, \lceil0.75\rceil, \lceil0.8\rceil\} = \{1,1,1\}&\\
                \therefore \hspace{0.5em} & f^{-1}(D) = \{1\}& 
            \end{flalign}
    \end{enumerate}
    
\item Let $ A = \{0,1\}. $ List the following relations:
    \begin{enumerate}[label=(\roman*)]
        \item List all the binary relations on $A$.
            \begin{flalign}\hspace{4em}
                & \{\{(0,0)\}, \{(0,1)\}, \{(1,0)\}, \{(1,1)\}, & \\
                & \{\{(0,0),(0,1)\}, \{(0,0),(1,0)\}, \{(0,0),(1,1)\}, \{(0,1),(1,0)\}, & \\
                & \{\{(0,1),(1,1)\}, \{(1,0),(1,1)\}, \{(0,0),(0,1),(1,0)\}, \{(0,0),(0,1),(1,1)\}, & \\
                & \{\{(0,0),(1,0),(1,1)\}, \{(0,1),(1,0),(1,1)\}, \{(0,0),(0,1),(1,0),(1,1)\}, \emptyset \} & 
            \end{flalign}
        \item List the reflexive relations on $A$.
            \begin{flalign}\hspace{4em}
                & \{\{(0,0),(1,1)\}, & \\ 
                & \{(0,0),(0,1),(1,1)\}, & \\
                & \{(0,0),(1,0),(1,1)\}, & \\
                & \{(0,0),(0,1),(1,0),(1,1)\}\} &
            \end{flalign}
        \item List the irreflexive relations on $A$.
            \begin{flalign}\hspace{4em}
                & \{\emptyset, \{(0,1)\}, \{(1,0)\}, \{(0,1),(1,0)\}\} &
            \end{flalign}
        \item List the symmetric relations on $A$.
            \begin{flalign}\hspace{4em}
                & \{\emptyset, \{(0,0)\}, \{(1,1)\}, & \\
                & \{(0,0),(1,1)\}, \{(0,1),(1,0)\}, & \\
                &\{(0,0),(0,1),(1,0)\}, \{(0,1),(1,0),(1,1)\}, & \\
                & \{(0,0),(0,1),(1,0),(1,1)\}, \{(0,0),(0,1),(1,0),(1,1)\}\} &
            \end{flalign}
        \item List the transitive relations on $A$.
            \begin{flalign}\hspace{4em}
                & \{\emptyset, \{(0,0)\}, \{(0,1)\},\{(1,0)\}, \{(1,1)\}, \{(0,0),(0,1)\},& \\
                & \{(0,0),(1,0)\}, \{(0,0),(1,1)\}, \{(0,1),(1,1)\}, & \\
                & \{(1,0),(1,1)\}, \{(0,0),(0,1),(1,1)\}, \{\{(0,0),(1,0),(1,1)\}\} &
            \end{flalign}
        \item List the antisymmetric relations on $A$.
            \begin{flalign}\hspace{4em}
                & \{\emptyset, \{(0,0)\}, \{(0,1)\},\{(1,0)\}, \{(1,1)\}, \{(0,0),(0,1)\},& \\
                & \{(0,0),(1,0)\}, \{(0,0),(1,1)\}, \{(0,1),(1,1)\}, & \\
                & \{(1,0),(1,1)\}, \{(0,0),(0,1),(1,1)\}, \{\{(0,0),(1,0),(1,1)\}\} &
            \end{flalign}
        \item List the asymmetric relations on $A$.
            \begin{flalign}\hspace{4em}
                & \{\emptyset, \{(0,1)\},  \{(1,0)\}\} &
            \end{flalign}
        \item List the relations on $A$ that are reflexive and symmetric.
            \begin{flalign}\hspace{4em}
                 & \{\{(0,0),(1,1)\}, & \\ 
                 & \{(0,0),(0,1),(1,0),(1,1)\}\} &
            \end{flalign}
        \item List the reflexive relations on $A$ that are neither reflexive nor irreflexive.
            \begin{flalign}\hspace{4em}
                & \{\{(0,0)\}, \{(1,1)\}, \{(0,0),(0,1)\}, \{(0,0),(1,0)\}, & \\ 
                & \{(0,1),(1,1)\}, \{(1,0),(1,1)\}, \{(0,0),(0,1),(1,0)\}, \{(0,1),(1,0),(1,1)\}\} &
            \end{flalign}
    \end{enumerate}
\end{enumerate}
\end{document}



