\documentclass[12pt]{article}
\usepackage[paper=letterpaper,margin=2cm]{geometry}
\usepackage{amsmath}
\usepackage{amssymb}
\usepackage{amsfonts}
\usepackage{newtxtext, newtxmath}
\usepackage{enumitem}
\usepackage{titling}
\usepackage[colorlinks=true]{hyperref}

\setlength{\droptitle}{-6em}

\title{Assignment \#4}
\author{Aidan Traboulay}
\date{\today}

\begin{document}
\maketitle
\begin{enumerate}[leftmargin=\labelsep]
\item Suppose that a "word" is any string of seven letters of the alphabet, with repeated letters allowed.
    \begin{enumerate}
        \item How many words begin with A or B and end with A or B?
            \begin{flalign}\hspace{4em}
                \nonumber & \text{Given the first and last letters begin with A or B} \to 2^2 & \\
                \nonumber & \text{Thus, there are 5 other letter} \to 26^5 & \\
                \nonumber & (2^2)\cdot(26^5) = 47525504  & \\
                 & \therefore 47525504 \hspace{0.4em} \text{combinations of words that begin with A or B and end with A or B} &
            \end{flalign}
        \item How many words begin with a vowel or end with a vowel?
            \begin{flalign}\hspace{4em}
                \nonumber & \text{Let A = combos beginning with vowel}: 5\cdot 26^6 & \\
                \nonumber & \text{Let B = combos ending with vowel}: 5\cdot 26^6 & \\
                \nonumber & \text{A \& B = combos beginning with and ending with vowel}: 5^2 \cdot 26^5 \\
                \nonumber & \therefore \text{Combos beginning or ending with vowel} : [(5\cdot26^6) + (5\cdot26^6)] - 5^2\cdot26^5 & \\
                & = 2792123360 \hspace{0.3em} \text{words} &
            \end{flalign}
    \end{enumerate}

\item Consider all bit strings of length 12.
    \begin{enumerate}
        \item How many begin with 110?
            \begin{flalign}\hspace{4em}
                \nonumber & 111 222 222 222 & \\ 
                & \therefore 1^3 \cdot 2^9 = 512 &
            \end{flalign}
        \item How many have exactly four 1’s?
            \begin{flalign}\hspace{4em}
                & C(12,4) = \frac{12!}{4!8!} = 495 &
            \end{flalign}
    \end{enumerate}

\item A club with 20 women and 17 men needs to form a committee of size six.
    \begin{enumerate}
        \item How many committees are possible if the committee must have three women and three men?
            \begin{flalign}\hspace{4em}
                \nonumber & \text{3 Men: C(17,3)} & \\
                \nonumber & \text{3 Women: C(20,3)} & \\
                & \therefore C(17,3) \cdot C(20,3) = \frac{17!}{3!14!} \cdot \frac{20!}{3!17!} = 775200 &
            \end{flalign}
        \item How many committees are possible if the committee must have at least two men?
            \begin{flalign}\hspace{4em}
                \nonumber & \text{Possible combinations: }& \\
                \nonumber & \text{Min: 2 men, 4 women:} \hspace{0.4em} C(17,2)\cdot C(20,4) + &  \\
                \nonumber & \text{3 men, 3 women:} \hspace{0.4em} C(17,3)\cdot C(20,3)  + & \\
                \nonumber & \text{4 men, 2 women:} \hspace{0.4em} C(17,4)\cdot C(20,2)  + & \\
                \nonumber & \text{5 men, 1 women:} \hspace{0.4em} C(17,5)\cdot C(20,1) + & \\
                \nonumber & \text{Max: 6 men, 0 women:} \hspace{0.4em} C(17,6)\cdot C(20,0) & \\
                \nonumber & = (136\cdot4845) + (680\cdot1140) + (2380\cdot190) + (6188\cdot20) + (12376\cdot1) & \\
                & = 2022456 &
            \end{flalign}
    \end{enumerate}

\item Use the binomial theorem to expand $(2c - 3d)^4.$
        \begin{flalign}\hspace{4em}
            \nonumber & \text{Binomial Theorem: } \frac{(a+b)}{n} = \sum_{k=0}^{n} C(n,k)\cdot a^{n-k}\cdot b^k& \\
            \nonumber & (2c -3d)^4 = \frac{(a+b)}{n} = \sum_{k=0}^{4} C(4,k)\cdot (2c)^{4-k}\cdot (-3d)^k & \\
            \nonumber & = C(4,0) \cdot (2c)^4 \cdot (-3d)^0 + C(4,1) \cdot (2c)^3 \cdot (-3d)^1+ C(4,2) \cdot (2c)^2 \cdot (-3d)^2 & \\
            \nonumber & + C(4,3) \cdot (2c)^1 \cdot (-3d)^3 + C(4,4) \cdot (2c)^0 \cdot (-3d)^4 & \\ 
            & = 16c^4 - 96c^3d + 216c^2d^2 - 216cd^3 + 81d^4 &
        \end{flalign}

\item 
    \begin{enumerate}
        \item What is the probability that a fair coin lands Heads 6 times in a row?
            \begin{flalign}\hspace{4em}
               \nonumber & \text{A fair coin for either heads or tails = } (\frac{1}{2}) & \\
               & \text{6 heads:} (\frac{1}{2})^6 = \frac{1}{64} \hspace{0.4em} \text{probability}&
            \end{flalign}
        \item What is the probability that a fair coin lands Heads 4 times out of 5 flips?
            \begin{flalign}\hspace{4em}
               \nonumber & \text{A fair coin for either heads or tails = } (\frac{1}{2}) & \\
                \nonumber & \text{A fair coin flips heads = } \frac{1}{16}& \\
                \nonumber & \text{Combination of 4 heads on 5 flips} = C(5,4) = 5 & \\ 
                & \therefore \frac{1}{2} \cdot \frac{1}{16} \cdot 5 = \frac{5}{32} \hspace{0.4em} \text{probability}&
            \end{flalign}
    \end{enumerate}
\vspace{9em}
\item Suppose you pick two cards, one at a time, at random, from an ordinary deck of 52 cards. Find
    \begin{enumerate}
        \item $ p $ (both cards are diamonds).
            \begin{flalign}\hspace{4em}
                \nonumber & \# \text{ of diamonds in 52-card deck:} 13 & \\
                \nonumber & \text{Let} \hspace{0.4em} e_0 \hspace{0.4em} \text{= 1st card being diamond} & \\
                \nonumber & \text{Let} \hspace{0.4em} e_1 \hspace{0.4em} \text{= 2nd card being diamond} & \\
                \nonumber & P(e_0) = \frac{13}{52} = \frac{1}{4} & \\
                \nonumber & P(e_1) = \frac{12}{52} = \frac{4}{17} & \\
                & P(e_0 \cdot e_1) = \frac{1}{4} \cdot \frac{4}{17} = \frac{1}{17}&
            \end{flalign}
        \item $ p $ (the cards form a pair).
            \begin{flalign}\hspace{4em}
                \nonumber & \forall \hspace{0.3em} \text{pairs there are 4 suites}: C(4,2) & \\
                \nonumber & C(52,2) \hspace{0.3em} \text{represents pairs for the 52-card deck} & \\
                \nonumber & \text{Since for each suite there are 13 cards} & \\
                \nonumber & \text{Let e = cards that form a pair} & \\
                & \therefore P(e) = \frac{13\cdot C(4,2)}{C(52,2)} = \frac{1}{17} &
            \end{flalign}
    \end{enumerate}
    
\item Suppose you have 40 different books (20 math books, 15 history books, and 5
geography books).
    \begin{enumerate}
        \item You pick two books at random, one at a time. What is the probability that both books are history books?
            \begin{flalign}\hspace{4em}
                \nonumber & \text{For history books}: C(15,2) = \frac{15!}{2!13! } & \\
                \nonumber & \text{For all books}: C(40,2) = \frac{40!}{2!38!} & \\
                & \therefore \frac{C(15,2)}{C(40,2)} = \frac{7}{52} \hspace{0.4em}\text{probability} &  
            \end{flalign}
        \item You pick two books at random, one at a time. What is the probability that the two books are from different disciplines?
            \begin{flalign}\hspace{4em}
                \nonumber & \text{Let A = A history book and a math book: } C(15,1) \cdot C(20,1) & \\
                \nonumber & \text{Let B = A history book and a geography book:} C(5,1) \cdot C(15,1) & \\
                \nonumber & \text{Let C = A geography book and a math book:} C(5,1) \cdot C(20,1)& \\
                & \therefore \frac{[P(A) + P(B) + P(C)]}{C(40,2)} = \frac{300 + 75 + 100}{780} = \frac{475}{780} = \frac{95}{156}&
            \end{flalign}
    \end{enumerate}
\vspace{9em}
\item Suppose you pick a bit string from the set of all bit strings of length ten.
    \begin{enumerate}
        \item What is the probability that the bit string has exactly two 1’s?
            \begin{flalign}\hspace{4em}
                \nonumber & \text{Possible 10-bit strings}: 2^{10} & \\
                \nonumber & \text{Possible 10-bit strings with two 1's}: C(10,2) & \\
                \nonumber & \text{Let E = exactly two 1's} & \\
                & \therefore P(E) = \frac{C(10,2)}{2^{10}} = \frac{45}{1024} &
            \end{flalign}
        \item What is the probability that the bit string has exactly two 1’s, given that the string begins with a 1?
            \begin{flalign}\hspace{4em}
                \nonumber & \text{Possible 10-bit strings}: 2^{10} & \\
                \nonumber & \text{Possible 10-bit strings starting at 1}: C(9,1) & \\
                \nonumber & \text{Let E = a string with exactly two 1's that begin with 1} & \\
                & \therefore P(E) = \frac{C(9,1)}{2^{10}} = \frac{9}{1024} &
            \end{flalign}
    \end{enumerate}
\end{enumerate}
\end{document}
